\chapter{Software and programs used in this thesis}  \label{appendix_software}
This PhD project required a vast amount of computing, preprocessing, analysis and plotting.
None of this would have been possible without the large number of different software packages used, all of which are free to use, and most of which are open source.
The following is a non-comprehensive list of the software used:
\begin{multicols}{2}
    \begin{description}
        \item[R] \citet{R}
        \item[CDO] \citet{CDO}
        \item[NCO] \citet{Zender2008}
        \item[Python] \citet{python}
        \item[netCDF] \citet{netcdf}
        \item[ScyPy] \citet{Jones2007}
        \item[GDAL] \citet{GDAL}
    \end{description}
\end{multicols}

Most of the data analysis and plotting was carried out using R. Several R packages were extremely useful and deserve a special mention:
\begin{multicols}{2}
    \begin{description}
 \item[ncdf4] \citet{ncdf4}
 \item[ggplot2] \citet{ggplot2}
 \item[patchwork] \citet{patchwork}
 \item[ggspatial] \citet{ggspatial}
% \item[ggalt] \citet{ggalt}
 \item[ggrepel] \citet{ggrepel}
 \item[RColorBrewer] \citet{RColorBrewer}
% \item[shinyjs] \citet{shinyjs}
% \item[shinydashboard] \citet{shinydashboard}
 \item[sf] \citet{sf}
 \item[stars] \citet{stars}
 \item[raster] \citet{raster}
 \item[rnaturalearth] \citet{rnaturalearth}
 \item[mapedit] \citet{mapedit}
 \item[leaflet] \citet{leaflet}
 \item[shiny] \citet{shiny}
 \item[dplyr] \citet{dplyr}
 \item[tidyr] \citet{tidyr}
 \item[glue] \citet{glue}
 \item[readr] \citet{readr}
 \item[profvis] \citet{profvis}
 \item[purrr] \citet{purrr}
 \item[furrr] \citet{furrr}
 \item[future] \citet{future}
 \item[futile.logger] \citet{futile.logger}
 \item[optparse] \citet{optparse}
 \item[lubridate] \citet{lubridate}
    \end{description}
\end{multicols}

This thesis was typeset in \LaTeX.
