\chapter{Introduction}

\section{Project objectives and overview}
The aim of this Thesis, partially funded by the Allianz Insurance Company, is to produce flood risk maps over the complete Italian domain for both the present day climate and for future projections. Due to the requirements of a strictly physically-based reproducible scientific approach, a framework consisting in a model chain of three tried-and-tested models, spanning climate, hydrology, and hydraulics, was developed. This Thesis thus describes a truly inter-disciplinary approach to flood risk modeling.\\
In order to obtain a reliable representation of flood risk, model calibration and evaluation was performed using several observational datasets of precipitation, discharge and flood extent. In particular, a new gridded hourly precipitation dataset for the complete Italian territory was developed in conjunction with the University of L'Aquila. The development of such dataset represents a necessary step in the scientific process described in this Thesis since, to our knowledge, no database suitable for driving a high-resolution hourly hydrological model is currently available over Italy.\\

\section{The structure of this Thesis}
DESCRIBE HOW IS THIS THESIS ORGANIZED

\section{Floods and inundation forecast and risk: an overview}
Floods are some of the most devastating natural disasters, with strong impacts on both the social and economic scale.

\section{The future of extreme climatological events}

\section{The climatological-hydrological-hydraulical approach}
