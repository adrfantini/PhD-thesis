\begin{abstract}
\addchaptertocentry{\abstractname} % Add the abstract to the table of contents
Floods are one of the most devastating natural disasters, with strong impacts on both society and economy.
Flood hazard estimation is an essential tool for protecting the population from floods both financially, via insurance policies, and physically, via water management and engineering.
In Europe, different kinds of flood maps are usually produced with different methods by governments, regional agencies, or insurance and re-insurance companies.
In the last decades, however, new approaches based on hydrological and hydraulical modelling emerged as viable, which allows for a more robust and reproducible physically-based estimation of flood extent and water level.
In this work a multimodel approach has been adopted, with a hydrological model  driven by multiple sources of precipitation input data, generating discharge data for a given region or basin for a long time period.
The discharge is then fed to a hydrodynamic model which produces of flood extents and, if necessary, other variables such as flood depth or flow speed.
Extreme value analysis is also used to derive any Return Period value (ranging from 10 to 500 years) starting from a shorter observational record, by assuming a given distribution for extreme events.

Over Italy, flood hazard has not yet been evaluated using a unified method.
We propose a methodology to simulate flood hazard for any return period, based on a model chain comprised of three models: the Regional Climate Model RegCM, the hydrological model CHyM and the hydraulical model CA2D.
For the time being only domains covering the complete Italian territory were simulated, but no limitation is in place that would prevent this technique from being applied to any domain worldwide, provided that the necessary input data is available.

In order to provide observed precipitation data as input to the hydrological model, we created a new product called GRIPHO (GRidded Italian Precipitation Hourly Observations), which consists of quality checked hourly precipitation observations over the complete Italian territory.
We validate GRIPHO against other state-of-the-art precipitation datasets over Italy, showing good performance in reproducing both mean and extreme precipitation: GRIPHO is comparable with the high resolution ARCIS and EURO4M-APGD datasets in Northern Italy, and shows finer spatial details and more consistent extremes than E-OBS in the rest of the domain.

Two regional climate simulations, one run in perfect boundary mode (with ERA-Interim) and one a scenario simulation (driven with  HadGEM, RCP8.5), are described and validated over the Italian territory.
Both simulations show good agreement with observation in several precipitation and temperature metrics, for both extreme and mean climate.
The projected climate change signal is also evaluated, finding, on average, increased extreme precipitation even in areas where mean precipitation decreases.

Three hydrological simulations, driven by both observations and regional climate  model outputs, are described.
Validation is carried out against a set of discharge stations, finding generally good performance of the CHyM model for the regions where observations are available.
Using different metrics, we assess the future changes in mean and extreme discharge for the Italian territory, finding strong increases in possible flood proxies.
In particular, mean discharge is projected to increase (decrease) in Northern Italy in winter (summer), which is directly linked to changes in mean precipitation over the Po river basin.
In winter and autumn, maximum yearly discharges increase by 50\% in several Italian regions, with summer and spring showing more mixed results.
100-year Return Period discharges are projected to increase over most of Italy by more than 100\% by the end of the century.
Similarly, the frequency of exceedance of extreme discharge thresholds drastically increases in the future scenario compared to the reference period, with changes for the 100-year threshold exceeding 500\% for several rivers in Italy, including the Po river.

Finally, we present preliminary results over the whole Italian territory for flood maps for different Return Periods, as produced by a hydraulic model fed by simulated discharge data.
The resulting maps are compatible with the currently available products obtained from regional environment agencies, but have the advantage of being produced with a coherent, scientific methodology.
\end{abstract}